\documentclass{report}

\usepackage[colorlinks=true, linkcolor=blue]{hyperref} %hyper references 

\title{Updating and Maintaining a Modded Minecraft Server Running Debian 8.3.0}
\author{Nicholas Rust}
\date{\today}

\begin{document}

\maketitle

%table of contents, centered with black links
{\hypersetup{linkcolor=black}
	\centering
	\tableofcontents
}

\pagebreak

This document assumes the following:
\begin{itemize}
	\item You have a Debian 8.3.0 system set up to run minecraft. 
	\item You are setting up a modded minecraft survival server.
	\item You have some command line experience.
\end{itemize}

\chapter{Updating Your Modded Server}

\section{Upload the Server Pack}
The following commands will:
\begin{enumerate}
	\item Install the unzip package. Skip if you already have it or an alternative. 
	\item Move newserverpack.zip to /opt/minecraft directory.
	\item Unzip the serverpack.
	\item Set the minecraft user as the owner of the directory.
\end{enumerate}

\begin{verbatim}
	sudo apt install unzip 
	mv newserverpack.zip /opt/minecraft/
	unzip /opt/minecraft/newserverpack.zip
	sudo chown -R minecraft:minecraft opt/minecraft/newserverpack/
\end{verbatim}

\section{Configure eula.txt}
Open eula.txt and change "false" to "true" indicating you have read the Mojang End User License Agreement and agree to it. The server will not start if this is not set to true.

\section{Delete session.lock}
Delete the file oldserverpack/world/session.lock if it exists.

This file locks the world, preventing the server from starting up using this world. This file may not always exist. If it does not exist it will not cause a problem.

\section{Copy All Modified Directories and Files}
Use the cp (copy) command to duplicate and place the old directories you wish to keep into the newserverpack.

\textbf{Remember:} Tab completion works with the following commands.

\begin{verbatim}
	cd /opt/minecraft/oldserverpack/
	cp -tr ../newserverpack/ world backups server.properties
	 banned-players.json banned-ips.json 
\end{verbatim}

\subsection{Copy Plugin Directories}
\begin{verbatim}
	cp -tr ../newserverpack/ luckperms/ nucleus/
\end{verbatim}

\subsection{Copy Plugin .jar Files}
These .jar files will generate all of their configs if this is the first time adding them to the server.

\subsubsection*{Move into the mods directory.}

\begin{verbatim}
	cd /opt/minecraft/oldserverpack/mods/
\end{verbatim}

\subsubsection*{Generic Example:}

\begin{verbatim}
	cp -tr ../../newserverpack/mods/ file1.jar file2.jar file3.jar 
\end{verbatim}

\subsubsection*{Real Example:}

\begin{verbatim}
	cp -tr ../../rebirth_of_the_night/mods/ 0spongeforge-Version.jar
	 CatClearLag-Version.jar FTBBackups-Version.jar griefprevention-Version.jar
	 LuckPerms-Version.jar Nucleus-Version.jar TotalEconomy-Version.jar
\end{verbatim}

\subsection{Copy Plugin Configuration Files}
We need these since we have edited them to suit our server and regenerating them based on the jar would revert them to default.

\subsubsection*{Move into the config directory.}

\begin{verbatim}
	cd /opt/minecraft/oldserverpack/config/
\end{verbatim}

\subsubsection*{Generic Example:}

\begin{verbatim}
	cp -tr ../../newserverpack/config directory1/ config1.cfg directory2/
\end{verbatim}

\subsubsection*{Real Example:}

\begin{verbatim}
	cp -tr ../../rebirth_of_the_night/config catclearlag/ ftbbackups.cfg
	 griefprevention/ luckperms/ nucleus/ sponge/ totaleconomy/
\end{verbatim}

\subsection{Update Plugins (optional)}
Since most server packs do not come with their plugins pre-installed you will have to update them yourself. This step is not always necessary but can be helpful in avoiding bugs. Download the new version of the .jar and replace the old one in the newserverpack/mods directory. It is still recommended to copy over the configs as their use seldomly changes between plugin versions.

\subsection{Configure server.properties}
Here is a link to the \href{https://minecraft.gamepedia.com/Server.properties}{minecraft wikipedia} (\url{https://minecraft.gamepedia.com/Server.properties}) page for the server.properties file. The key can be a great resource if you want to read up on specific options. Ideally nothing will need to change for a modpack update.

\subsection{Configure configs}
I have found config manipulation to be a big part of modded minecraft server management. Be familiar with the configs and spend some time looking through them. These vary so significantly by various modpack I cannot go into specifics in this document. Explore and do some tests.\\

\textbf{Note:} While you may be able to change these while the server is running, a server restart will be required before they take effect in game.

\subsubsection{Create a new directory with the new files}
Be sure not to just delete your old directory and throw the new one in, it is very likely we want to save some of the files.



\subsection{How to view the minecraft terminal using screen}

\begin{enumerate}
	\item sudo su minecraft   (to enter the minecraft user)
	\item screen -ls		  (list active screen IDNUMBERS and what they are linked to)
	\item screen -r IDNUMBER  (connect to a screen)
\end{enumerate}
 
 
\subsection{If your IP address changes}

sudo dig beartooth.bendy.cool

sudo ddclient


WORLD RESET
Nucleus userdata Homes
totaleconomy money
griefprevention claims

LAG FINDER
/debug start
wait time
/debug stop

sponge disable world

/spawnpoint (playername)  (x, y, z)
/give fergenbergel contenttweaker:spawn\_scroll



\subsection{World Wipe}

\subsection{No world wipe}

\end{document}