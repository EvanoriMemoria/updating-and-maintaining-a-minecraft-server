\documentclass{article}

\usepackage{hyperref}

\title{Updating and Maintaining a Minecraft Server}
\author{Nicholas Rust}
\date{\today}

\begin{document}

\maketitle

\tableofcontents

\pagebreak

\section{Basic Setup (Debian)}
For access to the minecraft terminals:
apt-get install screen 

\subsection{Create a minecraft group and user}

\subsection{Upload and Configure Server Files}

\subsubsection{eula.txt}
Open eula.txt and change false to true indicating you have read the Mojang eula and agree to it. The server will not start if this is not set to true.

\subsubsection{server.properties}

\subsubsection{configs}

\subsection{Port Forwarding and Firewall}

\subsubsection{Uncomplicated Firewall (ufw)}

sudo apt install ufw

For IPv6 go to:
sudo nano /etc/default/ufw

IPV6=yes\\

sudo ufw allow [port]

sudo ufw enable

More detailed information can be found here:
\url{https://www.digitalocean.com/community/tutorials/how-to-set-up-a-firewall-with-ufw-on-debian-9}

\subsubsection{SSH}
Using ufw you can allow ssh connections with the command

sudo ufw allow 22

\section{Updating to the Latest Version of a Modpack}

\subsection{World Wipe}

\subsection{No world wipe}

\end{document}